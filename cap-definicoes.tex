%% ------------------------------------------------------------------------- %%
\chapter{Definições}
\label{cap:definicoes}

Um grafo \(G = (V, A)\) é um par ordenado, onde \(V\) e \(A\) são conjuntos disjuntos e cada elemento de \(A\) é um par não-ordenado de elementos de \(V\). Os elementos de \(V\) são chamados de \textbf{vértices} e os de \(A\) são chamados de \textbf{arestas}.

Um grafo é dito \textbf{incremental} se suporta a adição de arestas. Analogamente, um grafo dinâmico é dito \textbf{decremental} se suporta a remoção de arestas. Um grafo incremental \textbf{ou} decremental é dito \textbf{parcialmente dinâmico}. Um grafo incremental \textbf{e} decremental é dito \textbf{totalmente dinâmico}.

