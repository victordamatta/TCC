% !TEX root = monografia.tex
\chapter{Ambiente}
\label{cap:ambiente}

O ambiente utilizado no trabalho foi o ELF (?).
Criado para a pesquisa em IA, sua escolha propõe múltiplos benefícios: 
proporciona todos os desafios característicos de jogos RTS,
dispensa das complexidades irrelevantes para pesquisa que estão presentes em jogos destinados para o mercado,
é integrado com uma framework de Aprendizado de Reforço, baseada em PyTorch (?),
faz um uso altamente eficiente dos recursos computacionais,
e, por último, proporciona alto grau de controle para o usuário.

O jogo consiste de dois jogadores 
disputando por recursos, 
construindo unidades 
e as controlando 
com o objetivo final de destruir a base do adversário.
Mais precisamente, 
cada jogador começa com sua base em cantos opostos de um mapa quadricular,
que contém fontes de recursos. 
É importante ressaltar que os jogadores só tem visão do que está suficiente próximo de suas unidades 
(mas independentemente disso, possuem conhecimento da localização da base inimiga).

O estado disponível para os algoritmos (respeitando a área de visão) é uma discretização do mapa em uma grid 20x20,
e a informação é distribuída em múltiplos canais, 
primeiro revelando a posição de unidades de cada tipo, 
depois o HP 
e por fim um canal que representa as fontes de recursos.

\begin{table}[]
    \begin{tabular}{|l|p{8cm}|}
    \hline
    Comando              & Descrição                                                                                       \\
    \hline
    INATIVO              & Não faz nada                                                                                    \\
    \hline
    CONSTRÓI-TRABALHADOR & Se a base está inativa, constrói um trabalhador.                                                \\
    \hline
    CONSTRÓI-QUARTEL     & Move um trabalhador (coletando ou inativo) para um lugar vazio e constrói um quartel.           \\
    \hline
    CONSTRÓI-GLADIADOR   & Se existe um quartel inativo, constrói um gladiador.                                            \\
    \hline
    CONSTRÓI-TANQUE      & Se existe um quartel inativo, constrói um tanque.                                               \\
    \hline
    BATER-E-CORRER       & Se existem tanques, move eles em direção a base inimiga e ataca, fuja se forem contra atacados. \\
    \hline
    ATACAR               & Gladiadores e tanques atacam a base inimiga.                                                    \\
    \hline
    ATACAR EM ALCANCE    & Gladiadores e tanques atacam inimigos a vista.                                                  \\
    \hline
    TODOS A DEFESA       & Gladiadores e tanques atacar tropas inimigas perto da base ou da fonte de recursos.             \\
    \hline
    \caption{Tabela de Ações}
    \end{tabular}
\end{table}

